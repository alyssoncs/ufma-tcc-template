\chapter{\foreign{Misc}}\label{chap:solucao}

\section{Código}

Inclusão de código pode ser feita através do pacote \href{https://ctan.org/pkg/listings}{listings}. Utilize conforme o \autoref{lst:factorial-inline}.

\lstinputlisting[
	float=htb, 
	style=latexStyle, 
	label=lst:factorial-inline, 
	caption=Inclusão de código
]{\CurrentFilePath/res/code/lstlisting.tex}

O conteúdo dentro do bloco \texttt{\textbackslash lstlisting} será colocado dentro do seu documento, como os exemplos de código que você viu até agora, no caso do \autoref{lst:factorial-inline}, este iria gerar como output o \autoref{lst:factorial-inline-output}.

\begin{lstlisting}[
    float=htb, 
    style=javaStyle, 
    caption={Fatorial},
    label=lst:factorial-inline-output
]
public class Main {
    public static void main(String[] args) {
        System.out.println(factorial(10));
    }
    
    static int factorial(int n) {
        if (n == 1) return 1;
        
        return n * factorial(n - 1);
    }
}
\end{lstlisting}

Caso queira deixar seu código \LaTeX{} mais compacto, você pode colocar o seu código em um arquivo separado e incluir em seu documento com o comando \texttt{\textbackslash lstinputlisting}. Esse \template sugere que coloque esse códigos em \contentdir{}\texttt{\{nome-da-seção\}/}\codedir. Assumindo que o código existe em um arquivo \texttt{Factorial.java} dentro desse diretório, o exemplo anterior poderia ser escrito conforme o \autoref{lst:factorial-include}.

\begin{lstlisting}[
    float=htb, 
    style=latexStyle, 
    caption={Inclusão de código através de um arquivo externo},
    label=lst:factorial-include
]
\lstinputlisting[
	float=htb, 
	style=javaStyle, 
	label=lst:factorial-include-output, 
	caption=Fatorial
]{\CurrentFilePath/res/code/Factorial.java}
\end{lstlisting}

\subsection{Linguagens e estilos}

Você pode especificar o estilo de código para uma determinada linguagem alterando o arquivo \pkgfile, incluindo cores e palavras reservadas adicionais, siga conforme os estilos já definidos. Consulte a documentação oficial para coisas mais complexas.

\section{Remissões internas} \label{sec:cross-ref}

Se você quer referenciar partes do seu documento, como: capítulos, seções, subseções, figuras, tabelas e códigos, você deve primeiramente adicionar uma \labelterm nesse elemento, depois deve usar o comando \texttt{\textbackslash autoref} ou \texttt{\textbackslash ref} no local que deseja mencioná-lo.

O comando \texttt{\textbackslash autoref} adiciona um link para o elemento contendo o nome que o identifica (como: \enquote{capítulo}, \enquote{seção}, \enquote{subseção}, \enquote{tabela} e etc.) seguido do número desse elemento no documento.

O comando \texttt{\textbackslash ref} adiciona um link para o elemento contendo \emph{apenas} o número do elemento no documento.

Como exemplo, adicionei a \labelterm \enquote{\texttt{sec:cross-ref}} na seção atual, ao utilizar o comando \texttt{\textbackslash autoref\{sec:cross-ref\}}, vou obter o seguinte resultado: \enquote{\autoref{sec:cross-ref}}. 
Entretanto, se utilizar \texttt{\textbackslash ref\{sec:cross-ref\}}, vou obter apenas: \enquote{\ref{sec:cross-ref}}, que pode ser útil ao referenciar múltiplos elementos ao mesmo tempo.

A forma de adicionar \labels é ligeiramente diferente dependendo do tipo de elemento sendo referenciado.

\subsection{\Labels em capítulos, seções e subseções}

Para capítulos, seções e subseções, usamos o comando \texttt{label} seguido do id da \labelterm que queremos atribuir, conforme o \autoref{lst:chap-sec-subsec-label}.

\begin{lstlisting}[
    float=htb, 
    style=latexStyle, 
    caption={Atribuição de \labels em capítulos, seções e subseções},
    label=lst:chap-sec-subsec-label,
]
\chapter{Introdução} \label{chap:intro}

Blá.

\section{Panorama Geral} \label{sec:overview}

Blá.

\subsection{Questões Fundamentais} \label{sub:fundamental-questions}

Blá.
\end{lstlisting}

Desta forma, eu posso referenciar o capítulo \enquote{Introdução} usando o comando: \texttt{\textbackslash autoref\{chap:intro\}}.

\subsection{\Labels em figuras}

Também se utiliza o comando \texttt{label}, entretanto, dentro do ambiente \texttt{figure} e após a macro \texttt{caption}, consulte o \autoref{lst:figure}.

\subsection{\Labels em tabelas}

Também se utiliza o comando \texttt{label}, entretanto, dentro do primeiro bloco do comando \texttt{IBGEtab} e após a macro \texttt{caption}, consulte o \autoref{lst:tab}.

\subsection{\Labels em códigos}

\emph{Não} utilizamos comando \texttt{label} nesse caso, utilizaremos o argumento opcional \texttt{label} presente no ambiente \texttt{lstlisting} e no comando \texttt{lstinputlisting}, consulte os códigos \ref{lst:factorial-inline} e \ref{lst:factorial-include}.

\section{Aspas}

Aspas em \LaTeX\ são mais complicadas do que parecem. Ao tentar usar os caracteres \enquote{\texttt{\textquotesingle}} ou \enquote{\texttt{"}} (aspas simples e dupla, respectivamente), o resultado não será o esperado, não gerando as aspas curvas.
O uso errôneo desse caracteres no \autoref{lst:wrong-quotation} irá gerar o seguinte resultado:
'Aspas simples' e "aspas duplas".

\begin{lstlisting}[
    float=htb, 
    style=latexStyle, 
    caption={Uso incorreto de aspas},
    label=lst:wrong-quotation,
]
(*@\textquotesingle@*)Aspas simples(*@\textquotesingle@*) e "aspas duplas".
\end{lstlisting}

Note que \enquote{\texttt{\textquotesingle}} sempre gera aspas direitas e \enquote{\texttt{"}} sempre gera aspas duplas retas.
Para produzir aspas corretamente, o \LaTeX\ espera que se utilize os símbolos presentes na \autoref{tab:quotes}.

\begin{table}[htb] 
	\begin{center} 
		\IBGEtab{
			\caption{Uso de aspas no \LaTeX}
			\label{tab:quotes}
		}{
			\begin{tabular}{lclc}
				\toprule
				Resultado desejado    & Símbolo                                   & Descrição do símbolo & Resultado \\
				\midrule 
				Aspa simples esquerda & \texttt{\`}                               & 1 acento grave       & ` \\ 
				Aspa simples direita  & \texttt{\textquotesingle}                 & 1 aspa simples       & ' \\ 
				Aspa dupla esquerda   & \texttt{\`{}\`{}}                         & 2 acentos grave      & `` \\ 
				Aspa dupla direita    & \texttt{\textquotesingle\textquotesingle} & 2 aspas simples      & '' \\ 
				\bottomrule 
			\end{tabular} 
		}{ 
			\fonte {\autoriapropria} 
		} 
	\end{center} 
\end{table} 

Ou seja, o exemplo anterior deveria ser escrito conforme o \autoref{lst:correct-quotation}, gerando:
`Aspas simples' e ``aspas duplas''.

\begin{lstlisting}[
    float=htb, 
    style=latexStyle, 
    caption={Uso correto de aspas},
    label=lst:correct-quotation,
]
(*@\`{}@*)Aspas simples(*@\textquotesingle@*) e (*@\`{}\`{}@*)aspas duplas(*@\textquotesingle\textquotesingle@*).
\end{lstlisting}

Como esse método de inclusão de aspas é pouco intuitivo e fácil de errar, podemos usar o pacote \href{https://ctan.org/pkg/csquotes}{csquotes} para auxiliar. Usamos os comando \texttt{enquote*} para aspas simples e \texttt{enquote} para aspas duplas.
O exemplo pode ser então reescrito conforme o \autoref{lst:csquotes}, gerando:
\enquote*{Aspas simples} e \enquote{Aspas duplas}.

\begin{lstlisting}[
    float=htb, 
    style=latexStyle, 
    caption={Uso do pacote csquotes},
    label=lst:csquotes,
]
\enquote*{Aspas simples} e \enquote{Aspas duplas}.
\end{lstlisting}

\section{Termos muito usados no seu documento}

Se o seu texto utiliza com bastante frequência termos que você quer que tenham tipografia e escrita consistente, elas podem ser adicionadas como macros no arquivo \macrofile, com o comando \texttt{newterm}. Vamos olhar alguns exemplos no \autoref{lst:newterm}.

\begin{lstlisting}[
    float=htb, 
    style=latexStyle, 
    caption={Definição de macros},
    label=lst:newterm,
]
\newterm{\api}{API}
\newterm{\broadcast}{\foreign{broadcast}}
\newterm{\printf}{\texttt{printf}}
\end{lstlisting}

Neste caso, foram definidos 3 macros (\texttt{\textbackslash api}, \texttt{\textbackslash broadcast} e \texttt{\textbackslash printf}), a intenção do autor é que o termo \api sempre seja escrito em maiúsculo, o termo \broadcast sempre seja escrito como uma palavra estrangeira e o termo \printf sempre seja formatado com fonte monoespaçada.

Para usar os macros definidos pelo autor, basta incluir no texto desejado, conforme o \autoref{lst:macro-usage}, gerando o seguinte resultado:

Usando essa \api, você pode fazer um \broadcast de dados para vários dispositivos. Para verificar o status, pode-se utilizar \printf para imprimir mensagens no console.

\begin{lstlisting}[
    float=htb, 
    style=latexStyle, 
    caption={Uso de macros},
    label=lst:macro-usage,
]
Usando essa \api, você pode fazer um \broadcast de dados para 
vários dispositivos. Para verificar o status, pode-se utilizar 
\printf para imprimir mensagens no console.
\end{lstlisting}

\section{Unidades de media}

Aconselho o uso do pacote \href{https://ctan.org/pkg/siunitx}{siunitx}, a leitura do manual é extremamente útil para descobrir todos os comandos que você vai precisar usar. Aqui está um resumo:

\subsection{Unidades}

Use \texttt{unit} para escrever unidades de medida. Esse comando aceita duas sintaxes, uma usando macros predefinidas, e outra usando strings. Por exemplo, para escrever \enquote{metros por segundo ao quadrado}, podemos usar \verb|\unit{\meter\per\second\squared}| ou \verb|\unit{m/s^2}|. Entretanto, esses dois comandos vão produzir resultados levemente diferente, conforme

\begin{table}[htb] 
	\begin{center} 
		\IBGEtab{
			\caption{Especificando unidades de media}
			\label{tab:unit}
		}{
			\begin{tabular}{lc}
				\toprule
				Comando                                                                                                           & Resultado \\
				\midrule 
                \texttt{\textbackslash unit\{\textbackslash meter\textbackslash per\textbackslash second\textbackslash squared\}} & \unit{\meter\per\second\squared} \\
                \texttt{\textbackslash unit\{m/s\textasciicircum 2\}}                                                             & \unit{m/s^2} \\ 
				\bottomrule 
			\end{tabular} 
		}{ 
			\fonte {\autoriapropria} 
		} 
	\end{center} 
\end{table} 

Apesar de representar a mesma unidade, por padrão, o macro  \verb|\per| usa o formato de exponencial, isso pode ser alterado passando um argumento opcional, da seguinte forma: \verb|\unit[per-mode=symbol]{\meter\per\second\squared}|, gerando: \unit[per-mode=symbol]{\meter\per\second\squared}.

Também existem alguns macros abreviados, no caso anterior, poderíamos escrever
\verb|\unit[per-mode=symbol]{\m\per\s\squared}|, para obter: \unit[per-mode=symbol]{\m\per\s\squared}.

Os prefixos como quilo, mega, e giga também estão definidos e podem ser usados normalmente, para obter  
\unit{\kilo\gram}, usa-se \verb|\unit{\kilo\gram}|. 

Prefixos binários também estão disponíveis (kibi, mebi, gibi, etc), assim como macros para bit e byte, por exemplo: \verb|\unit[per-mode=symbol]{\mibi\byte\per\second}|, produz \unit[per-mode=symbol]{\kibi\byte\per\second}.

\subsection{Quantidades}

Quantidades podem ser especificadas com a macro \texttt{qty}, essa macro aceita dois parâmetros obrigatórios, o número e a unidade. Para especificar \enquote{5 gibibytes}, você usaria \verb|\qty{5}{\gibi\byte}|, que produz: \qty{5}{\gibi\byte}.

\section{Fonte}

A \abnt não determina um tipo de fonte específico, mas é um equívoco comum que deve ser utilizado a fonte Arial. Se você precisar mudar a fonte, esse documento é compilado com \XeLaTeX\ então você pode remover o pacote fontenc e usar o pacote \href{https://ctan.org/pkg/fontspec}{fontspec} e usar qualquer fonte em seu computador.

