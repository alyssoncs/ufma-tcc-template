\setlength{\absparsep}{18pt} % ajusta o espaçamento dos parágrafos do resumo

\begin{resumo}
	Com a crescente demanda por formatos padronizados e a importância de facilitar o processo de formatação, a utilização de templates em \LaTeX\ para a elaboração de monografias tem se tornado uma prática comum em diversas instituições acadêmicas.
	No contexto da \ufma, a adoção de um \template específico para monografias, alinhado ao padrão \abnt, visa garantir a padronização da estrutura e a conformidade com as normas da instituição e da \abntfull.
	Esse template facilita a organização dos conteúdos e a inserção de elementos como referências bibliográficas, tabelas e figuras.
	Este trabalho tem como objetivo apresentar e detalhar o uso de um template em \LaTeX\ para a elaboração de monografias na \ufma, abordando suas principais funcionalidades e vantagens no processo de formatação conforme as diretrizes da \abnt.

	\textbf{Palavras-chave}: \abnt. \ufma.
\end{resumo}

\begin{resumo}[Abstract]
	With the growing demand for standardized formats and the importance of simplifying the formatting process, the use of \LaTeX\ templates for writing theses has become a common practice in various academic institutions.
	In the context of \ufma, the adoption of a specific template for theses, aligned with \abnt standards, aims to ensure the standardization of structure and compliance with the institution's and the Brazilian Association of Technical Standards (\abnt) guidelines.
	This template helps organize content and insert elements such as bibliographic references, tables, and figures.
	This work aims to present and detail the use of a \LaTeX\ template for writing theses at \ufma, highlighting its main features and advantages in the formatting process according to \abnt guidelines.

	\textbf{Keywords}: \abnt. \ufma.
\end{resumo}
