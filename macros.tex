% Aplica formatação a palavras estrangeiras com o comando \foreign
% \foreign{<palavra>}
% Este comando aplica formatação em itálico à palavra fornecida, que é ideal para termos ou expressões estrangeiras.
% 
% Exemplo de uso:
% \foreign{dataset}
% Dentro do documento, "dataset" será exibido em itálico.
\newcommand*{\foreign}[1]{\textit{#1}}

% Define um novo termo com o comando \newterm
% \newterm{<nome do comando>}{<termo>}
% Este comando cria um novo comando de nome <nome do comando>, que representa o termo <termo>.
% O comando pode ser usado dentro do corpo do documento e será substituído pelo termo definido.
%
% Exemplo de uso:
% \newterm{\iot}{IoT}
% Agora, dentro do documento, ao usar \iot, ele será substituído por IoT.
% Você também pode definir um termo que inclua formatações extra. Exemplo:
% \newterm{\importantnote}{\textbf{Nota importante}} % negrito
% \newterm{\println}{\texttt{printf}} % monoespaçado
\newcommand*{\newterm}[2]{\newcommand*{#1}{#2\xspace}}

% Defina aqui os termos comumente usados no seu documento onde você quer garantir uma escrita e/ou formatação consistente; Sinta-se livre para organizar como bem entender.
% Algumas palavras estrangeiras são consideradas comuns o suficiente de forma que é aceito e aconselhável que não sejam grafadas com itálico, não existe uma regra, use o bom senso. Você também pode seguir o manual de comunicação da SECOM como referência: https://www12.senado.leg.br/manualdecomunicacao/estilos/estrangeirismos-grafados-sem-italico-ou-aspas

% Meta
\newterm{\mainfile}{\texttt{monografia.tex}}
\newterm{\metadatafile}{\texttt{metadata.tex}}
\newterm{\macrofile}{\texttt{macro.tex}}
\newterm{\pkgfile}{\texttt{tccconfig.sty}}
\newterm{\contentdir}{\texttt{content/}}
\newterm{\bibfile}{\texttt{bib/biblio.bib}}
\newterm{\codedir}{\texttt{res/code}}

% Conceitos
\newterm{\abntex}{abntex2}
\newterm{\template}{\foreign{template}}
\newterm{\labelterm}{\foreign{label}}
\newterm{\labels}{\foreign{labels}}
\newterm{\Label}{\foreign{Label}}
\newterm{\Labels}{\foreign{Labels}}
\newterm{\api}{API}
\newterm{\broadcast}{\foreign{broadcast}}
\newterm{\dataset}{\foreign{dataset}}

% Substantivos próprios
\newterm{\abnt}{ABNT}
\newterm{\abntfull}{Associação Brasileira de Normas Técnicas}
\newterm{\ufma}{UFMA}

% Tecnologias
\newterm{\mqtt}{MQTT}
\newterm{\printf}{\texttt{printf}}

% Utils
\newterm{\autoriapropria}{Produzido pelo autor}

